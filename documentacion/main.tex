\documentclass[12pt, letterpaper]{article}
\usepackage[utf8]{inputenc}

\title{Proyecto final: propuesta.}
\author{Kevin Esguerra Cardona y Juan Esteban García}
\date{\today}

\begin{document}
\maketitle

En la sociedad actual, se hace cada vez más importante tener buenas contraseñas. No hablo de contraseñas como:
\begin{itemize}
   \item 1003499387
   \item Juan12
\end{itemize}
Hablo de contraseñas que fidelicen la seguridad del usuario en todas las plataformas que usa. Ante esta problemática, nos encontramos dos situaciones.

En un caso, tenemos una contraseña muy bien pensada, con letras mayús- culas, minúsculas, números y caracteres especiales; todo repartido de manera uniforme. Pero, para no olvidarla la usamos en todas partes. O peor aún, se le dejamos al gestor de contraseñas de nuestro navegador favorito.

Por otra parte, existe un tipo de personas que se preocupan por el hecho de dejar al descubierto toda su información personal con perder una sola llave. Entonces, tienen cientos de contraseñas igual de complicadas que la del ejemplo anterior. Pero ahora, se ven ante un dilema gigante. Y, es que es humanamente imposible, al menos para el promedio, recordar más de 2 o 3 contraseñas de este tipo. Y, las personas en internet suelen visitar 2 o 3 sitios web diferentes cada 10 minutos, en promedio. Eso quiere decir, que un usuario estándar de internet, para garantizar su seguridad, debería memorizar entre 12 y 18 contraseñas para navegar en internet de forma efectiva durante un par de horas. También, se puede dar el caso de que este usuario, en su afán por no olvidar sus credenciales, decida almacenarlas en un archivo de texto plano (.txt). O peor aún, dejárselas al gestor de contraseñas.

Pensando en esto, no hemos propuesto construir una solución. Como proyecto final, crearemos un programa que le permita al usuario almacenar sus contraseñas de forma local y segura.

El usuario podrá gestionar todo sobre sus contraseñas. Crear nuevas, actualizar o borrar viejas. Y, todo de forma segura. Sin la posibilidad de exfiltración de datos. Cabe aclarar que las contraseñas podrán ser ingresadas por el usuario o generadas por el programa.

Aparte de eso, el usuario podrá obtener una lista de sus contraseñas, la cual podrá personalizar por medio de filtros: según el correo, el nombre de usuario, la empresa o sitio web al que pertenece, etc.

También, y para terminar este primer documento, decir que el usuario podrá pedirle al programa que almacene la contraseña en el portapapeles y que lo redirija al sitio web o aplicación que le corresponde.
\end{document}